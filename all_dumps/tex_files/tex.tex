$ d(((x)^5+(3)^x \cdot 81)) = d((x)^5) + d((3)^x \cdot 81)$

Посчитаем составные части:

$ d((3)^x \cdot 81) = d((3)^x) \cdot (81) + d(81) \cdot ((3)^x)$

Посчитаем составные части:

$ d(81) = 0$

$ d((3)^x) = x \cdot (3)^(x - 1) \cdot d(3)$

Посчитаем составные части:

$ d(3) = 0$

Получилось:

$ d((3)^x) = 0$

Получилось:

$ d((3)^x \cdot 81) = 0$

$ d((x)^5) = 5 \cdot (x)^(5 - 1) \cdot d(x)$

Посчитаем составные части:

$ d(x) = 1$

Получилось:

$ d((x)^5) = 5 \cdot (x)^4$

Получилось:

$ d(((x)^5+(3)^x \cdot 81)) = 5 \cdot (x)^4$

