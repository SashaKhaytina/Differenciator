$ d(((x)^5+(x)^3 \cdot 81)) = d((x)^5) + d((x)^3 \cdot 81)$

Посчитаем составные части:

$ d((x)^3 \cdot 81) = d((x)^3) \cdot (81) + d(81) \cdot ((x)^3)$

Посчитаем составные части:

$ d(81) = 0$

$ d((x)^3) = 3 \cdot (x)^(3 - 1) \cdot d(x)$

Посчитаем составные части:

$ d(x) = 1$

Получилось:

$ d((x)^3) = 3 \cdot (x)^2$

Получилось:

$ d((x)^3 \cdot 81) = 3 \cdot (x)^2 \cdot 81$

$ d((x)^5) = 5 \cdot (x)^(5 - 1) \cdot d(x)$

Посчитаем составные части:

$ d(x) = 1$

Получилось:

$ d((x)^5) = 5 \cdot (x)^4$

Получилось:

$ d(((x)^5+(x)^3 \cdot 81)) = (5 \cdot (x)^4+3 \cdot (x)^2 \cdot 81)$

