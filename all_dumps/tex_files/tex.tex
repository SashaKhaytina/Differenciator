\documentclass{article}
\usepackage{graphicx}
\usepackage[russian]{babel}
\title{Differenciator}
\begin{document}
\maketitle
\section{Возьмем эту производную}


$ d(((((x)^{5}+(x)^{3 \cdot x})-ln(sin(x)) \cdot 81)+\frac{cos(x)}{54})) = d((((x)^{5}+(x)^{3 \cdot x})-ln(sin(x)) \cdot 81)) + d(\frac{cos(x)}{54})$

Посчитаем составные части:

$ d(\frac{cos(x)}{54}) = \frac{d(cos(x)) \cdot (54) + d(54) \cdot (cos(x))}{54}$

Посчитаем составные части:

$ d(54) = 0$

$ d(cos(x)) = (-1) \cdot sin(x) \cdot d(x)$

Посчитаем составные части:

$ d(x) = 1$

Получилось:

$ d(cos(x)) = -1 \cdot sin(x)$

Получилось:

$ d(\frac{cos(x)}{54}) = \frac{-1 \cdot sin(x) \cdot 54}{2916}$

$ d((((x)^{5}+(x)^{3 \cdot x})-ln(sin(x)) \cdot 81)) = d(((x)^{5}+(x)^{3 \cdot x})) - d(ln(sin(x)) \cdot 81)$

Посчитаем составные части:

$ d(ln(sin(x)) \cdot 81) = d(ln(sin(x))) \cdot (81) + d(81) \cdot (ln(sin(x)))$

Посчитаем составные части:

$ d(81) = 0$

$ d(ln(sin(x))) = \frac{d(sin(x))}{sin(x)}$

Посчитаем составные части:

$ d(sin(x)) = cos(x) \cdot d(x)$

Посчитаем составные части:

$ d(x) = 1$

Получилось:

$ d(sin(x)) = cos(x)$

Получилось:

$ d(ln(sin(x))) = \frac{cos(x)}{sin(x)}$

Получилось:

$ d(ln(sin(x)) \cdot 81) = \frac{cos(x)}{sin(x)} \cdot 81$

$ d(((x)^{5}+(x)^{3 \cdot x})) = d((x)^{5}) + d((x)^{3 \cdot x})$

Посчитаем составные части:

$ d((x)^{3 \cdot x}) = (x)^{3 \cdot x} \cdot d(ln(x) \cdot (3 \cdot x))$

Посчитаем составные части:

$ d(ln(x) \cdot 3 \cdot x) = d(ln(x)) \cdot (3 \cdot x) + d(3 \cdot x) \cdot (ln(x))$

Посчитаем составные части:

$ d(3 \cdot x) = d(3) \cdot (x) + d(x) \cdot (3)$

Посчитаем составные части:

$ d(x) = 1$

$ d(3) = 0$

Получилось:

$ d(3 \cdot x) = 3$

$ d(ln(x)) = \frac{d(x)}{x}$

Посчитаем составные части:

$ d(x) = 1$

Получилось:

$ d(ln(x)) = \frac{1}{x}$

Получилось:

$ d(ln(x) \cdot 3 \cdot x) = (\frac{1}{x} \cdot 3 \cdot x+ln(x) \cdot 3)$

Получилось:

$ d((x)^{3 \cdot x}) = (x)^{3 \cdot x} \cdot (\frac{1}{x} \cdot 3 \cdot x+ln(x) \cdot 3)$

$ d((x)^{5}) = (x)^{5} \cdot d(ln(x) \cdot (5))$

Посчитаем составные части:

$ d(ln(x) \cdot 5) = d(ln(x)) \cdot (5) + d(5) \cdot (ln(x))$

Посчитаем составные части:

$ d(5) = 0$

$ d(ln(x)) = \frac{d(x)}{x}$

Посчитаем составные части:

$ d(x) = 1$

Получилось:

$ d(ln(x)) = \frac{1}{x}$

Получилось:

$ d(ln(x) \cdot 5) = \frac{1}{x} \cdot 5$

Получилось:

$ d((x)^{5}) = (x)^{5} \cdot \frac{1}{x} \cdot 5$

Получилось:

$ d(((x)^{5}+(x)^{3 \cdot x})) = ((x)^{5} \cdot \frac{1}{x} \cdot 5+(x)^{3 \cdot x} \cdot (\frac{1}{x} \cdot 3 \cdot x+ln(x) \cdot 3))$

Получилось:

$ d((((x)^{5}+(x)^{3 \cdot x})-ln(sin(x)) \cdot 81)) = (((x)^{5} \cdot \frac{1}{x} \cdot 5+(x)^{3 \cdot x} \cdot (\frac{1}{x} \cdot 3 \cdot x+ln(x) \cdot 3))-\frac{cos(x)}{sin(x)} \cdot 81)$

Получилось:

$ d(((((x)^{5}+(x)^{3 \cdot x})-ln(sin(x)) \cdot 81)+\frac{cos(x)}{54})) = ((((x)^{5} \cdot \frac{1}{x} \cdot 5+(x)^{3 \cdot x} \cdot (\frac{1}{x} \cdot 3 \cdot x+ln(x) \cdot 3))-\frac{cos(x)}{sin(x)} \cdot 81)+\frac{-1 \cdot sin(x) \cdot 54}{2916})$

\end{document}
